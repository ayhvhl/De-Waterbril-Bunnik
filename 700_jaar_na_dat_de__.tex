700 jaar na dat de Romeinen vertrokken begon de bevolking in Europa te groeien waardoor er meer voedsel verbouwd moest gaan worden.
Door het ontginnen van kleigronden verder van de rivieren werd de grondwaterstand verlaagt en begon deze in te klinken.
In de hoger gelegen gebieden werden de veengronden ook geschikt gemaakt voor landbouw wat leidde tot oxidatie en inklinking.
In de 11de eeuw was dit proces al zo ver gevorderd dat sommige delen bij vloed onderliepen.

In 1122 kreeg Utrecht ook zijn stadsrechten.
Dit omdat de Bisschop van Utrecht meer land wilde ontginnen.
Het project omvatte een dam bij Wijk bij Duurstede, om het grondwater te verlagen,
en een kanaal tussen Nieuwegein en Utrecht om de handel door te laten gaan.
Dit vonden de Utrechters een veel te duur plan en kwamen in opstand.
De Duitse keizer die toen ons land regeerde koos de kant van de burgers.
Na het gevecht met de Bisschop verleende hij Utrecht stadsrechten als dank voor de hulp van de burgers.
\footnote{\url{https://www.youtube.com/watch?v=V37I9He-kO8&list=PL70A1F712A8684111&index=4}}


