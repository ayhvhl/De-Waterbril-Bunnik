700 jaar na dat de Romeinen vertrokken begon de bevolking in Europa te groeien waardoor er meer voedsel verbouwd moest gaan worden.
Door het ontginnen van kleigronden verder van de rivieren werd de grondwaterstand verlaagt en begon deze in te klinken.
In de hoger gelegen gebieden werden de veengronden ook geschikt gemaakt voor landbouw wat leidde tot oxidatie en inklinking.
In de 11de eeuw was dit proces al zo ver gevorderd dat sommige delen bij vloed onderliepen.

In 1122 kreeg Utrecht ook zijn stadsrechten.
Dit omdat de Bisschop van Utrecht meer land wilde ontginnen.
Het project omvatte een dam bij Wijk bij Duurstede, om het grondwater te verlagen,
en een kanaal tussen Nieuwegein en Utrecht om de handel door te laten gaan.
Dit vonden de Utrechters een veel te duur plan en kwamen in opstand.
De Duitse keizer die toen ons land regeerde koos de kant van de burgers.
Na het gevecht met de Bisschop verleende hij Utrecht stadsrechten als dank voor de hulp van de burgers.
\footnote{\url{https://www.youtube.com/watch?v=V37I9He-kO8&list=PL70A1F712A8684111&index=4}}

De klein schalige projecten werden steeds groter en het organiseren van het beheer van de dijken en dammen werd steeds moeilijker.
Uiteindelijk werd on 1255 het eerste waterschap opgezet, het Hoogheemraadschap Rijnland.
Het is ook door de Graaf van Holland erkent.
Het werd bestuurd door de groot land bezitters in het beheerde gebied.

Door veenafgraving in het westen van Nederland ontstonden grote watervlakten die bij storm en ontij leidde tot afkalving van het land er omheen.
De stad Reimerswaal is hier door verloren gegaan.
Om te voorkomen dat het nog harder mis ging is er sterk toezicht op de vervening gekomen.

In de 16de eeuw waren de bemalingstechnieken zo ver gevorderd dat de veen plassen droog gelegd werden.
Het Achtermeer is hier het eerste voorbeeld van.
De techniek werd ook steeds meer verbeterd en in de 19de eeuw worden de eerste stoomgemalen geïntroduceerd.

Na de watersnood ramp tijdens de eerste wereldoorlog in het Zuiderzee gebied word besloten om deze af te sluiten om er een zoet water meer van te maken.
Voor de tweede wereldoorlog is het project af.
Nu heeft Nederland een groter waterbuffer tot zijn beschikking om de landbouw van water te voorzien.

De watersnoodramp in 1953 was de aanleiding tot het bouwen van de delta werken.
In zeeland werd de kust ingekort van $700 km$ tot ongeveer $50 km$.

In het binnen land werden de rivieren ook langzaam ingeperkt door dijken.
En omdat de bevolking groeide werd er ook mee land droog gelegt voor gebruik in de landbouw.
Dit zorgde voor inklinking en oxidatie wat het maaiveld deed dalen.

Na de Franse tijd is de Hollandse waterlinie aan gelegd om de Fransen buiten de deur te houden.
De rivieren werden ingezet als defensie.
Om dat goed te doen was het nodig om de hoeveelheid water in de rivieren te reguleren.

Langzaam aan werd het beheer van het water meer georganiseerd en grootschaliger.
We kregen meer een meer invloed op de hoeveelheid en de snelheid van het water en waar het naartoe ging.
In 1798 is Rijkswaterstaat opgericht als onderdeel van het miniserie van binnenlandse zaken.
Na de ramp van 1809 werd er het "Comit\`e Central du Waterstaat" opgericht.

Het comité is veel bezig geweest met het bevaarbaar houden van de rivieren voor de handel.
De Rijn is gekanaliseerd en er zijn verschillende kanalen gegraven voor handel en het ontginnen en transport van turf.

Na dat het IJsselmeer gevormd is word er water naar toe gestuurd om ervoor te zorgen dat wij altijd genoeg zoet water hebben.

Het voedsel tekort tijdens de tweede wereldoorlog legde een grote nadruk op voedsel voorziening binnen ons eigen land.
Grootscheepse ontginningen en de ruilverkaveling zorgden er voor dat er meer voedsel geproduceerd werd maar ook voor meer daling van het land.

Verschillende droogtes deden ons beseffen dat we nog lang niet genoeg wisten om er voor te zorgen dat de oogst altijd binnen zou komen.
Er werd verder onderzocht en aangepast in ons water stelsel.

De kwaliteit van het water is iets waar we eigenlijk nog maar net mee bezig zijn.
Nu iets meer dan een eeuw.
We wilden goed water maar meestal werden vervuilingsproblemen opgelost door een groter meer te zoeken om de vervuiling te verdunnen.
Iedereen die stroom afwaarts woonde had over het algemeen pech.

In de vroege 20ste eeuw werd er begonnen met het aanleggen van riool systemen en er werd ge\"expirimenteerd met septictanks.
Het waterschap Dommel kreeg als eerste de taak om rioolwater te zuiveren in 1950.
De eerste riool zuiverings installatie in nederland werd in 1960 gebouwd.
Rond 1970 verschoof het opschonen van riool water van het zelfreinigend vermogen van de natuur naar het behandelen van riool water in een fabriek door een open brief van de toenmalig Minister Drees van Rijkswaterstaat.

Nu dat wij onze zaken op orde begonnen te krijgen merkten we ook dat het belangrijk was om er voor te zorgen dat de gebieden stroom opwaarts hun zaken op orde hebben.
Vanuit duitsland kwamen soms grote hoeveelheden water onze kant op.
En Frankrijk en duitsland vervuilden het water wat onze kant uit kwam.
In 1950 is er een verdrag getekend door alle landen waar de Rijn en de Maas door stromen, het IRC, over het beheer van deze rivieren.

Nu dat er veel minder geloosd werd in de rivieren zou het wel goed komen dacht men maar al het oude vuil moest nog opgeruimd worden. 
Vooral in de Rotterdamse havens was veel vervuiling dat bij elke onderhoudsbeurt weer vrij kwam als het opgebaggerd werd.

%Maak een tijdsbalk (van 0 tot nu) waarop je de watergeschiedenis van jouw woonomgeving weergeeft. Geef op deze tijdsbalk minimaal 5 belangrijke gebeurtenissen in de watergeschiedenis van jouw gebied weer en verduidelijk deze gebeurtenissen middels een korte toelichting en enkele foto’s, afbeeldingen of kaarten. Concentreer je hierbij op de invloed van de mens op het watersysteem. Richt je dus met name op aspecten als de aanleg van de eerste dijken, het vergraven van rivieren of beken, dijkdoorbraken en overstromingen, het gebruik van water in verdedigingslinies, het ontstaan van droogmakerijen en waterschappen, ruilverkavelingen, etc.

%Wat zie je nu nog terug van deze watergeschiedenis in jouw woonomgeving? Zijn er bepaalde elementen in het landschap die nu nog herinneren aan de watergeschiedenis? Geef hiervan een korte beschrijving, en maak gebruik van eigen gemaakte foto’s.
