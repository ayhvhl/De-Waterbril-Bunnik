\section{De locatie.}
Bunnik ligt ten zuidoosten van Utrecht aan de krommerijn. 
In de gemeente liggen naaast Bunnik ook de dorpen Odijk, Werkhoven en Vechten. 
Er wonen iets minder dan 15.000 mensen.\footnote{bron: \url{https://nl.wikipedia.org/wiki/Bunnik_(gemeente)}}

\section{ligging van het gebied.}
Bunnik ligt vlak naast Utrecht.
Net ten westen liggen de forten Rijnauwen en Vechten die bijde onderdeel waren van de hollandse waterlinie.
Het waterbeheer valt onder Hoogheemraadschap de Stichtse Rijnlanden.\footnote{\url{http://hdsr.nl}}

De gemiddelde hoogte van het naaiveld binnen bunnik ligt op ongeveer $+3.00 m$ NAP.
Net buiten de bebouwing daat deze naar ongeveen $+1.50 m$.
De kromme rijn loopt langs de noordkant van Bunnik langs Amelisweerd en Rijnouen naar Utrecht.

Er is relatief wijning oppervlakte water in de gemeente.
De Krommerijn loopt langs te noord-oost kant ven de gemeente.
De Hakswetering, die naar Zeist lijd, mond net boven bunnik uit in de kromme-rijn.
Rijnauwen en Amelisweerd zijn belangrijke recreatie gebienden aan de Zuidkant van Utrecht.
En delen zijn \ldots beschermt landschap \footnote het stukje met de heggen}

De omgeving is vrij vlak en er word voornamelijk veetelt bedraven.
Dit komt omdat het grondwater over het algemeen vrij hoog staat en de bodem, volgens mij, erg klei\"ig is.
Hier door heeft de bodem niet genoeg draagkracht om mashienes toe te staan op het land.
Op de iets hoger gelegen delen is de bodem wat zandiger wat de mogelijkheid bied tot boomgaarden en maisteelt.

%TODO Kaart ahn.
%In de mediatheek en op internet kun je allerlei kaarten vinden. De belangrijkste die wij nu gaan gebruiken zijn de topografische kaarten. Zoek een kaart met voldoende schaalgrootte (bijvoorbeeld 1:25.000) waarop ook de hoogtelijnen staan aangegeven. Geef vervolgens antwoord op de volgende vragen:

%In welk gedeelte van Nederland woon je? Ligt het gemiddelde maaiveld (landoppervlak) boven of onder NAP? En de omliggende omgeving?
%Ligt het in een overwegend natte omgeving of meer in een droog gebied? Indien nat dan staat het grondwater hoog en is er veel oppervlaktewater in de omgeving, indien droog dan is het omgekeerde het geval.
%Zonder het op te zoeken op de bodemkaart, heb je enig idee wat de grondsoort is van je omgeving? Betreft dit zand; klei; zware klei of veen?
%Is het gebied vlak of meer hellend of glooiend.
%Is het gebied voornamelijk stedelijk of landelijk? En wat is het landgebruik (akkers, weiland, natuur, bos, industrie, wonen, etc.)?

%Verduidelijk je beschrijving aan de hand van enkele foto’s.

