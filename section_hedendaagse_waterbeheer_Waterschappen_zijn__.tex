\section{hedendaagse waterbeheer}
%Waterschappen zijn in Nederland de organisaties die het waterbeheer grotendeels uitvoeren. Beantwoord onderstaande vragen om kennis te maken met de taken van jouw waterschap.
%Zoek de website van jouw waterschap (zie opdracht 1) op en bekijk deze website goed. Geef met behulp van de informatie op de website antwoord op de volgende vragen:

%Hoe groot is het waterschap?
Oppervlakte = 82.000 ha \footnote{\url{https://nl.wikipedia.org/wiki/Hoogheemraadschap_De_Stichtse_Rijnlanden}}
%Wat is het werkgebied (kaart)? pluk van wikipedia

%Hoeveel en welke gemeenten vallen onder het waterschap?
%http://www.hdsr.nl/werk/werkgebied/
Het werkgebied van HDSR \ldots

%Hoeveel gemalen, stuwen, watergangen, rioolwaterzuiveringsinstallaties, e.d. worden beheerd door het waterschap (kaart)? 
%Wat zijn de belangrijkste taken van het waterschap?
%Wie betaalt er allemaal waterschapslasten en hoeveel?
%Welke belangrijke projecten heeft het waterschap het afgelopen jaar uitgevoerd (bekijk minimaal drie projecten en geef een beschrijving per project in woord en beeld)?