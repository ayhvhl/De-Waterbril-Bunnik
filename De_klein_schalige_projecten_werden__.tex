De klein schalige projecten werden steeds groter en het organiseren van het beheer van de dijken en dammen werd steeds moeilijker.
Uiteindelijk werd on 1255 het eerste waterschap opgezet, het Hoogheemraadschap Rijnland.
Het is ook door de Graaf van Holland erkent.
Het werd bestuurd door de groot land bezitters in het beheerde gebied.

Door veenafgraving in het westen van Nederland ontstonden grote watervlakten die bij storm en ontij leidde tot afkalving van het land er omheen.
De stad Reimerswaal is hier door verloren gegaan.
Om te voorkomen dat het nog harder mis ging is er sterk toezicht op de vervening gekomen.

In de 16de eeuw waren de bemalingstechnieken zo ver gevorderd dat de veen plassen droog gelegd werden.
Het Achtermeer is hier het eerste voorbeeld van.
De techniek werd ook steeds meer verbeterd en in de 19de eeuw worden de eerste stoomgemalen geïntroduceerd.

In het binnen land werden de rivieren ook langzaam ingeperkt door dijken.
En omdat de bevolking groeide werd er ook mee land droog gelegt voor gebruik in de landbouw.
Dit zorgde voor inklinking en oxidatie wat het maaiveld deed dalen.

Na de Franse tijd is de Hollandse waterlinie aan gelegd om de Fransen buiten de deur te houden.
De rivieren werden ingezet als defensie.
Om dat goed te doen was het nodig om de hoeveelheid water in de rivieren te reguleren.

