Verschillende droogtes deden ons beseffen dat we nog lang niet genoeg wisten om er voor te zorgen dat de oogst altijd binnen zou komen.
Er werd verder onderzocht en aangepast in ons water stelsel.

De kwaliteit van het water is iets waar we eigenlijk nog maar net mee bezig zijn.
Nu iets meer dan een eeuw.
We wilden goed water maar meestal werden vervuilingsproblemen opgelost door een groter meer te zoeken om de vervuiling te verdunnen.
Iedereen die stroom afwaarts woonde had over het algemeen pech.

In de vroege 20ste eeuw werd er begonnen met het aanleggen van riool systemen en er werd ge\"expirimenteerd met septictanks.
Het waterschap Dommel kreeg als eerste de taak om rioolwater te zuiveren in 1950.
De eerste riool zuiverings installatie in nederland werd in 1960 gebouwd.
Rond 1970 verschoof het opschonen van riool water van het zelfreinigend vermogen van de natuur naar het behandelen van riool water in een fabriek door een open brief van de toenmalig Minister Drees van Rijkswaterstaat.

Nu dat wij onze zaken op orde begonnen te krijgen merkten we ook dat het belangrijk was om er voor te zorgen dat de gebieden stroom opwaarts hun zaken op orde hebben.
Vanuit duitsland kwamen soms grote hoeveelheden water onze kant op.
En Frankrijk en duitsland vervuilden het water wat onze kant uit kwam.
In 1950 is er een verdrag getekend door alle landen waar de Rijn en de Maas door stromen, het IRC, over het beheer van deze rivieren.

Nu dat er veel minder geloosd werd in de rivieren zou het wel goed komen dacht men maar al het oude vuil moest nog opgeruimd worden. 
Vooral in de Rotterdamse havens was veel vervuiling dat bij elke onderhoudsbeurt weer vrij kwam als het opgebaggerd werd.

%Maak een tijdsbalk (van 0 tot nu) waarop je de watergeschiedenis van jouw woonomgeving weergeeft. Geef op deze tijdsbalk minimaal 5 belangrijke gebeurtenissen in de watergeschiedenis van jouw gebied weer en verduidelijk deze gebeurtenissen middels een korte toelichting en enkele foto’s, afbeeldingen of kaarten. Concentreer je hierbij op de invloed van de mens op het watersysteem. Richt je dus met name op aspecten als de aanleg van de eerste dijken, het vergraven van rivieren of beken, dijkdoorbraken en overstromingen, het gebruik van water in verdedigingslinies, het ontstaan van droogmakerijen en waterschappen, ruilverkavelingen, etc.

%Wat zie je nu nog terug van deze watergeschiedenis in jouw woonomgeving? Zijn er bepaalde elementen in het landschap die nu nog herinneren aan de watergeschiedenis? Geef hiervan een korte beschrijving, en maak gebruik van eigen gemaakte foto’s.