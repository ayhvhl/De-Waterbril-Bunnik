Langzaam aan werd het beheer van het water meer georganiseerd en grootschaliger.
We kregen meer een meer invloed op de hoeveelheid en de snelheid van het water en waar het naartoe ging.
In 1798 is Rijkswaterstaat opgericht als onderdeel van het miniserie van binnenlandse zaken.
Na de ramp van 1809 werd er het "Comit\`e Central du Waterstaat" opgericht.

Het comité is veel bezig geweest met het bevaarbaar houden van de rivieren voor de handel.
De Rijn is gekanaliseerd en er zijn verschillende kanalen gegraven voor handel en het ontginnen en transport van turf.


Na de watersnood ramp tijdens de eerste wereldoorlog in het Zuiderzee gebied word besloten om deze af te sluiten om er een zoet water meer van te maken.
Voor de tweede wereldoorlog is het project af.
Nu heeft Nederland een groter waterbuffer tot zijn beschikking om de landbouw van water te voorzien.

De watersnoodramp in 1953 was de aanleiding tot het bouwen van de delta werken.
In zeeland werd de kust ingekort van $700 km$ tot ongeveer $50 km$.


Na dat het IJsselmeer gevormd is word er water naar toe gestuurd om ervoor te zorgen dat wij altijd genoeg zoet water hebben.

Het voedsel tekort tijdens de tweede wereldoorlog legde een grote nadruk op voedsel voorziening binnen ons eigen land.
Grootscheepse ontginningen en de ruilverkaveling zorgden er voor dat er meer voedsel geproduceerd werd maar ook voor meer daling van het land.

