\section{Watergeschiedenis}
%Tegenwoordig speelt water voor een deel een heel andere rol dan in onze vroegere historie. Nog maar een paar honderd jaar geleden ging al het vervoer in Nederland over water, water werd gebruikt voor de verdediging tegen de vijand, maar water was zelf ook een grote vijand (denk aan overstromingen, vloedgolven en dijkdoorbraken), water zorgde voor energie en al ons afval werd moeiteloos gedumpt in grachten en sloten. Uit die tijd stammen de waterpoorten, de grachten om de steden, waterlinies, terpen, droogmakerijen, watermolens de eerste dijken en dergelijke, maar ook de eerste waterschappen ontstonden in deze tijd.

%Maak ter inleiding een korte beschrijving (max 2 A4-tjes) van de watergeschiedenis van Nederland, vanaf de eerste ontginningen in West-Nederland tot nu, waarbij je de drie regio’s (laag Nederland, rivierengebied en hoog Nederland) aan bod laat komen, in woord en beeld. Gebruik hiervoor o.a. het dictaat op Blackboard en de presentatie van het college Watergeschiedenis. 

%Maak een tijdsbalk (van 0 tot nu) waarop je de watergeschiedenis van jouw woonomgeving weergeeft. Geef op deze tijdsbalk minimaal 5 belangrijke gebeurtenissen in de watergeschiedenis van jouw gebied weer en verduidelijk deze gebeurtenissen middels een korte toelichting en enkele foto’s, afbeeldingen of kaarten. Concentreer je hierbij op de invloed van de mens op het watersysteem. Richt je dus met name op aspecten als de aanleg van de eerste dijken, het vergraven van rivieren of beken, dijkdoorbraken en overstromingen, het gebruik van water in verdedigingslinies, het ontstaan van droogmakerijen en waterschappen, ruilverkavelingen, etc.

%Wat zie je nu nog terug van deze watergeschiedenis in jouw woonomgeving? Zijn er bepaalde elementen in het landschap die nu nog herinneren aan de watergeschiedenis? Geef hiervan een korte beschrijving, en maak gebruik van eigen gemaakte foto’s.
