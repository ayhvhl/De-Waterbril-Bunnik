\section{Watergeschiedenis}
%Tegenwoordig speelt water voor een deel een heel andere rol dan in onze vroegere historie. Nog maar een paar honderd jaar geleden ging al het vervoer in Nederland over water, water werd gebruikt voor de verdediging tegen de vijand, maar water was zelf ook een grote vijand (denk aan overstromingen, vloedgolven en dijkdoorbraken), water zorgde voor energie en al ons afval werd moeiteloos gedumpt in grachten en sloten. Uit die tijd stammen de waterpoorten, de grachten om de steden, waterlinies, terpen, droogmakerijen, watermolens de eerste dijken en dergelijke, maar ook de eerste waterschappen ontstonden in deze tijd.

%Maak ter inleiding een korte beschrijving (max 2 A4-tjes) van de watergeschiedenis van Nederland, vanaf de eerste ontginningen in West-Nederland tot nu, waarbij je de drie regio’s (laag Nederland, rivierengebied en hoog Nederland) aan bod laat komen, in woord en beeld. Gebruik hiervoor o.a. het dictaat op Blackboard en de presentatie van het college Watergeschiedenis. 
Nederland voor het jaar 800 was erg dun bevolkt.
Men woonde op de natuurlijke verhogingen van waar er gevist en gejaagt kon worden.
De opbrengsten daarvan werden aangevult met kleinschalige landbouw.
\footnote{Dictaat}
De Romeinen hebben ten noorden van Utrecht daarom om ook wijnig gebouwd.
\footnote{\url{https://nl.wikipedia.org/wiki/Traiectum_%28Utrecht%29#cite_ref-Montforts_14-0}}
\footnote{\url{https://www.youtube.com/watch?v=9ATMSGEu9R4}}

700 jaar na dat de Romeinen vertrokken begon de bevolking in europa te groeien waardoor er meer voedsel verbouwd moest gaan worden.
Door het ontginnen van kleigronden verder van de rivieren werd de grondwaterstand verlaagt en begon deze in te klinken.
In de hoger gelegen gebieden werden de veengronden ook geschikt gemaakt voor landbouw wat leide tot oxtidatie en inklinking.
In de 11de eeuw was dit proces al zo ver gevorderd dat sommige delen bij vloed onderliepen.

De klein schalige projecten werden steeds groter en het organiseren van het beheer van de dijken en dammen werd steeds moeilijker.
Uiteindelijk werd on 1255 het eerste waterschap opgezet, het Hoogheemraadschap Rijnland.
Het is ook door de Graaf van Holland erkent.
Het werd bestuurd door de groot land bezitters in het beheerde gebied.

Door veenafgraving in het westen van nederland ontstonden grote watervlakten die bij storm en ontij leide tot afkalving van het land er omheen.
De stad Reimerswaal is hier door verloren gegaan.
Om te voorkomen dat het nog harder mis ging is er sterk toezicht op de vervening gekomen.

In de 16de eeuw waren de bemalings technieken zo ver gevorderd dat de veen plassen droog gelegt werden.
Het Achtermeer is hier het eerste voorbeeld van.
De techniek werd ook steeds meer verbeterd en in de 19e eeuw worden de eerste stoom gemalen geintroduceerd.

Na de watersnood ramp tijdens de eerste wereldoorlog in het zuiderzee gebied word besloten om deze af te sluiten om er een zoet water meer van te maken.
Vlak voor de tweede wereloorlog is het project af.
Nu heeft Nederland een groter waterbuffer tot zijn beschikking om de landbouw van water te voorzien.


%Maak een tijdsbalk (van 0 tot nu) waarop je de watergeschiedenis van jouw woonomgeving weergeeft. Geef op deze tijdsbalk minimaal 5 belangrijke gebeurtenissen in de watergeschiedenis van jouw gebied weer en verduidelijk deze gebeurtenissen middels een korte toelichting en enkele foto’s, afbeeldingen of kaarten. Concentreer je hierbij op de invloed van de mens op het watersysteem. Richt je dus met name op aspecten als de aanleg van de eerste dijken, het vergraven van rivieren of beken, dijkdoorbraken en overstromingen, het gebruik van water in verdedigingslinies, het ontstaan van droogmakerijen en waterschappen, ruilverkavelingen, etc.

%Wat zie je nu nog terug van deze watergeschiedenis in jouw woonomgeving? Zijn er bepaalde elementen in het landschap die nu nog herinneren aan de watergeschiedenis? Geef hiervan een korte beschrijving, en maak gebruik van eigen gemaakte foto’s.
