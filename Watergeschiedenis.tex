\section{Watergeschiedenis}
%Tegenwoordig speelt water voor een deel een heel andere rol dan in onze vroegere historie. Nog maar een paar honderd jaar geleden ging al het vervoer in Nederland over water, water werd gebruikt voor de verdediging tegen de vijand, maar water was zelf ook een grote vijand (denk aan overstromingen, vloedgolven en dijkdoorbraken), water zorgde voor energie en al ons afval werd moeiteloos gedumpt in grachten en sloten. Uit die tijd stammen de waterpoorten, de grachten om de steden, waterlinies, terpen, droogmakerijen, watermolens de eerste dijken en dergelijke, maar ook de eerste waterschappen ontstonden in deze tijd.

%Maak ter inleiding een korte beschrijving (max 2 A4-tjes) van de watergeschiedenis van Nederland, vanaf de eerste ontginningen in West-Nederland tot nu, waarbij je de drie regio’s (laag Nederland, rivierengebied en hoog Nederland) aan bod laat komen, in woord en beeld. Gebruik hiervoor o.a. het dictaat op Blackboard en de presentatie van het college Watergeschiedenis. 
Nederland voor het jaar 800 was erg dun bevolkt.
Men woonde op de natuurlijke verhogingen van waar er gevist en gejaagd kon worden.
De opbrengsten daarvan werden aangevuld met kleinschalige landbouw.
\footnote{Dictaat}
De Romeinen hebben ten noorden van Utrecht daarom om ook weinig gebouwd.
\footnote{\url{https://nl.wikipedia.org/wiki/Traiectum_%28Utrecht%29#cite_ref-Montforts_14-0}}
\footnote{\url{https://www.youtube.com/watch?v=9ATMSGEu9R4}}
Maar ze hebben er wel een fort gebouwd, bij vechten, met een haven.
Deze is van 24 voor Christus tot 270 na Christus in gebruik geweest.

